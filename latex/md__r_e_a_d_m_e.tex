\chapter{traffic\+\_\+light\+\_\+system }
\hypertarget{md__r_e_a_d_m_e}{}\label{md__r_e_a_d_m_e}\index{traffic\_light\_system@{traffic\_light\_system}}
\label{md__r_e_a_d_m_e_autotoc_md0}%
\Hypertarget{md__r_e_a_d_m_e_autotoc_md0}%


This is a final project for the class CS410 Intro to Software Engineering at UMass Boston

To run and connect everything to the server\+:
\begin{DoxyItemize}
\item The arduino, device running the server, and the device running the flutter app have to be on the same network
\item Run the server via the command "{}node ./server"{} in the project directory. You have to have node js installed
\item After running the server, it should display its ip address and port in the terminal
\item Open the file Arduino/traffic\+\_\+light.\+ino
\begin{DoxyItemize}
\item Change the wifi ssid and password to your wifi\textquotesingle{}s name and password
\item Change the IP and port to the one displayed in the terminal from earlier
\end{DoxyItemize}
\item Open the file lib.\+main.\+dart
\begin{DoxyItemize}
\item Under the init\+State function, change the ip address and port to the one displayed in the terminal by the server
\end{DoxyItemize}
\item Compile and upload the arduino code. This should display it\textquotesingle{}s ip in the serial monitor.
\item On the device hosting the server, ping this ip by running the command "{}ping \{ip\}"{} 
\end{DoxyItemize}